\definecolor{Negro2}{HTML}{44475A}
\definecolor{Negro}{HTML}{282A36}
\definecolor{Blanco}{HTML}{F8F8F2}
\definecolor{Blanquisimo}{HTML}{FFFFFF}
\definecolor{Cian}{HTML}{003399}
\definecolor{Verde}{HTML}{006600}
\definecolor{Naranja}{HTML}{CC6600}
\definecolor{Rosa}{HTML}{CC66CC}
\definecolor{Purpura}{HTML}{660099}
\definecolor{Rojo}{HTML}{990000}
\definecolor{Amarillo}{HTML}{FFCC00}
\definecolor{Comentario}{HTML}{6272A4}

\newcommand{\s}{\hspace*{2em}}
\newcommand{\scc}[1]{\textrm{\textsc{#1}}}
\newcommand{\ra}{\rightarrow}
\newcommand{\Ra}{\Rightarrow}
\newcommand{\sii}{\Longleftrightarrow}
%\newcommand{\qed}{\null\hfill $\blacksquare$}
\newcommand{\tng}{\triangle}
\newcommand{\R}{\mathbb{R}}
\newcommand{\N}{\mathbb{N}}
\newcommand{\Dom}{\text{Dom}}
\newcommand{\Cod}{\text{Cod}}
\newcommand{\nvec}[1]{\vec{\mathbf{#1}}}
\newcommand{\uvec}[1]{\hat{\mathbf{#1}}}
\newcommand{\ninf}[1]{\norm{\nvec{#1}}_\infty}
\newcommand{\codeclass}[1]{\textrm{\textit{\textcolor{Rosa}{#1}}}}
\newcommand{\codevar}[1]{\textrm{\textit{\textcolor{Cian}{#1}}}}
\newcommand{\codemethod}[1]{\small{\textit{\textrm{\textcolor{Verde}{#1}}}}}
\newcommand{\codeString}[1]{\textcolor{Amarillo}{#1}}

% Traducción de nombres para algoritmos
\floatname{algorithm}{Algoritmo}
\renewcommand{\algorithmicrequire}{\textbf{Entrada:}}
\renewcommand{\algorithmicensure}{\textbf{Salida:}}

% Formatos para construcciones de teorema
\newtheorem{definition}{Definición}
\newtheorem{theorem}{Teorema}
\newtheorem{lemma}[theorem]{Lema}
%\theoremstyle{remark}
%\newtheorem*{remark}{A notar}
%\newtheorem*{idea}{Idea}

\lstdefinestyle{mystyle}{
    backgroundcolor=\color{Blanco},   
    commentstyle=\color{Negro},
    keywordstyle=\color{Rosa},
    numberstyle=\tiny\color{Purpura},
    stringstyle=\color{Amarillo},
    basicstyle=\ttfamily\footnotesize,
    breakatwhitespace=false,         
    breaklines=true,                 
    captionpos=b,                    
    keepspaces=true,                 
    numbers=left,                    
    numbersep=5pt,                  
    showspaces=false,                
    showstringspaces=false,
    showtabs=false,                  
    tabsize=4,
    numbers=none
}

% Diseño <<fancy>>
\fancyhf{}

% Encabezado
\lhead[]{ % izquierda
    \begin{tabular}{c c}
        \includegraphics[height=.25in]{resources/Logo_FC.png} & {\Facultad \hspace{1em} \UNIV} \\
    \end{tabular}
}

\rhead[]{ % derecha
    \begin{tabular}{c c}
        {\textit{\materia}} & \includegraphics[height=.25in]{resources/Logo_UNAM.png} \\
    \end{tabular}
}
%Pie de página

\cfoot[]{ % center
    {\color{Comentario}
    \begin{tabular}{c c c}
        $\boldsymbol{-}$ & \thepage & $\boldsymbol{-}$
    \end{tabular}
  }
}
